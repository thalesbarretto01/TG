\chapter{INTRODUÇÃO}
\label{chap:introducao}

Edite e coloque aqui o seu texto de introdução.

A Introdução é a parte inicial do texto, na qual devem constar o tema e a delimitação do assunto tratado, objetivos da pesquisa e outros elementos necessários para situar o tema do trabalho, tais como: justificativa, procedimentos metodológicos (classificação inicial), embasamento teórico (principais bases sintetizadas) e estrutura do trabalho, tratados de forma sucinta. Recursos utilizados e cronograma são incluídos quando necessário. Salienta-se que os procedimentos metodológicos e o embasamento teórico são tratados, posteriormente, em capítulos próprios e com a profundidade necessária ao trabalho de pesquisa.

\section{LEIA ESTA SEÇÃO ANTES DE COMEÇAR}
\label{sec:antesleiame}

Este documento é um \emph{template} \LaTeX{} que foi concebido, primariamente, para ser utilizado na elaboração de Trabalho de Conclusão de Curs em conformidade com as normas da Universidade Tecnológica Federal do Paraná.

Para a produção deste \emph{template} foi necessário adaptar o arquivo {\ttfamily abntex2.cls}. Assim, foi produzido o arquivo {\ttfamily utfpr-abntex2.cls} que define o \verb|documentclass| específico para a UTFPR.

Antes de começar a escrever o seu trabalho acadêmico utilizando este \emph{template}, é importante saber que há dois arquivos que você precisará editar para que a capa e a folha de rosto de seu trabalho sejam geradas automaticamente.
São eles os arquivos {\ttfamily capa.tex} e {\ttfamily folha-rosto.tex}, ambos no diretório  {\ttfamily /elementos-pre-textuais}.
No arquivo {\ttfamily capa.tex} deverá ser informado nome do autor, título do trabalho, natureza do trabalho, nome do orientador e outras informações necessárias.
No arquivo {\ttfamily folha-rosto.tex}, que contém o texto padrão estabelecendo que este documento é um requisito parcial para a obtenção do título pretendido, será necessário apenas comentar as linhas que não se aplicam ao tipo de trabalho acadêmico.

A compilação para gerar um arquivo no formato pdf, incluindo corretamente as referências bibliográficas, deve ser realizada utilizando o comando \verb|makefile|, disponível na mesma pasta onde está o arquivo principal \verb|utfpr-tcc.tex|. Caso seja alterado o nome do arquivo \verb|utfpr-tcc.tex|, deverá ser alterado no arquivo \verb|makefile| também.

\section{ORGANIZAÇÃO DO TRABALHO}
\label{sec:organizacaoTrabalho}

Normalmente ao final da introdução é apresentada, em um ou dois parágrafos curtos, a organização do restante do trabalho acadêmico.
Deve-se dizer o quê será apresentado em cada um dos demais capítulos.
